\documentclass{article}
\usepackage{amsmath}
\title{Homework 1}
\author{Aidan Gardner-O'Kearny}
\date{January 28, 2026}

\newcommand{\p}[1]{\left(#1\right)}
\newcommand{\braket}[1]{\langle#1\rangle}

\begin{document}
\maketitle

\section*{1)}

For a differentail cross section

\begin{align}
    \frac{d\sigma}{dt}=\p{\frac{ak_n}{m_n}}^{2}\frac{\pi}{|t|}
\end{align}

On the range $10^{-6}\leq|t|\leq{10^{-2}}$ is given by;

\begin{align*}
    \sigma&=\int_{10^{-6}}^{10^{-2}}\frac{d\sigma}{dt}dt
\end{align*}

\section*{2)}
For the fixed-target reaction $\pi^{-}p\rightarrow{K^0}\Lambda^{0}$, the pion beam is required to have a momentum of

\begin{align*}
    \sqrt{s}&=m_K+m_\Lambda\\[1em]
    &=0.497611\text{GeV}+1.115683\text{GeV}\\[1em]
    &=1.613294\text{GeV}\\[1em]
    s&=2.602718\text{GeV}
\end{align*}

To move to the fixed target frame;

\begin{align*}
    s&=\p{p^{\mu}_{\pi}}^2-\p{p^{\mu}_{P}}^2\\[1em]
    &=\p{E_{\pi}+m_P}^2-
\end{align*}

\section*{3)}
For a a particle with $M\gg{m_e}$, the maximum kinetic energy that can be transfered to an electron is given by


\section*{4)}

In proton-proton scattering, the maximum laboratory angle between the two particles will be $90$ degrees. This is because
the two particles are of equal mass. In the reletavistic limit, 


\section*{5)}
For an antineutrino detection process ($\bar\nu_{e}+{\rightarrow{e^+}+n}$) there is one parameter that determines interaction kinematics.
Choosing this parameter to be the 

\end{document}